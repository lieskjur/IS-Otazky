\documentclass{article}
\usepackage[utf8]{inputenc}

% Basic packages
	\usepackage{amssymb}
	\usepackage{amsmath}
	\usepackage{graphicx}
	\usepackage[czech]{babel}
	\usepackage{natbib}

% Relevant packages
	\usepackage{bm}
	\usepackage{bbm}
	\usepackage{geometry}
	\usepackage{enumitem}
	\usepackage{listings}
	\usepackage{multicol,float}

% Document settings
	\geometry{a4paper,margin=15mm}

	\setlist[itemize]{nolistsep,noitemsep}
		
	\setlength{\parindent}{0pt}
	\setlength{\parskip}{1.5em}
	\renewcommand{\baselinestretch}{1.2}

	\setlength{\abovedisplayskip}{1.2em}
	\setlength{\belowdisplayskip}{1.2em}
	\renewcommand{\arraystretch}{1.2}

\begin{document}
% \begin{multicols}{2}
	\section{Popište možné tvary dynamických systémů pro modely: spojité-diskrétní, lineární-nelineární. }
	
	\subsection*{Lineární spojitý systém v časové oblasti}
	\begin{align*}
		\bm{\dot{x}} &= \bm{A}\bm{x} + \bm{B}\bm{u} \\
		\bm{y} &= \bm{C}\bm{x} + \bm{D}\bm{u}
	\end{align*}
	\subsection*{Lineární spojitý systém ve frekvenční oblasti}
	\begin{align*}
		s\bm{x} &= \bm{A}\bm{x} + \bm{B}\bm{u} \\
		\bm{y} &= \bm{C}\bm{x} + \bm{D}\bm{u}
	\end{align*}
	Po úpravě $\bm{x}(s\bm{I} - \bm{A}) = \bm{B}\bm{u}$ můžeme dosadit
	\begin{equation}
		\bm{y} = \bm{C}(s\bm{I}-\bm{A})^{-1}\bm{B}\bm{u}+\bm{D}\bm{u}
	\end{equation}
	\subsubsection*{Dynamická poddajnost}
	\begin{equation}
	\bm{G}(\bm{x}) = \frac{\bm{y}}{\bm{u}} = \bm{C}(s\bm{I}-\bm{A})^{-1}\bm{B}+\bm{D}
	\end{equation}
	\subsection*{Lineární diskrétní systém}
	\begin{align*}
		\bm{x}_{t+\Delta t} &= \bm{M}\bm{x}_t + \bm{N}\bm{u}_t \\
		\bm{y}_t &= \bm{O}\bm{x}_t + \bm{P}\bm{u}_t
	\end{align*}
	diskrétní tvar lze získat z tvaru spojitého modelu v časové oblasti
	\begin{align*}
		\bm{M} = e^{\bm{A}\Delta t} \doteq \bm{I} + \bm{A}\Delta t
		\,,\;
		\bm{N} = \bm{B}
		\,,\;
		\bm{O} = \bm{C}
		\,,\;
		\bm{P} = \bm{D}
	\end{align*}
	\subsection*{Nelineární systém}
	\begin{align}
		\bm{\dot{x}} = \bm{f}(\bm{x}) + \bm{g}(\bm{x})\bm{u} \\
		\bm{y} = \bm{c}(\bm{x}) + \bm{d}(\bm{x}) \bm{u}
	\end{align}

	\section{Popište postup identifikace lineárního modelu SISO systému ve tvaru ARX a OE. }
	
	\subsection{Auto Regresive model with eXogenous inputs}
	Lineární filtr používajíci minulé vstupy a výstupy systému
	\begin{equation}
		\hat{y}(t) + a_1 y(t-\Delta t) + \dots + a_n y(t-n\Delta t)
		=
		b_0 u(t) + \dots + b_n u(t-n\Delta t)
	\end{equation}
	Pro $N$ kroků měření lze vypsat $N-n$ rovnic
	\begin{align*}
		\hat{y}_n &= -a_1 y_{n-1} - \dots - a_n y_0 + b_0 u_n + \dots + b_n u_0 \\
		&\vdots \\
		\hat{y}_N &= -a_1 y_{N-1} - \dots - a_n y_{N-n} + b_0 u_N + \dots + b_n u_{N-n}
	\end{align*}
	Ty lze zapsat ve tvaru
	\begin{equation}
	\bm{\hat{y}} = \bm{\Phi} \bm{p}
	\;,\quad 
	\end{equation}
	kde
	\begin{align*}
		\bm{\hat{y}} &= \begin{bmatrix} \hat{y}_n & \dots & \hat{y}_N \end{bmatrix}^T
		\\ 
		\bm{\Phi}
		&=
		\begin{bmatrix}
			y_{n-1} & \dots & y_0 & u_n & \dots & u_0 \\
			\vdots & \ddots & \vdots & \vdots & \ddots & \vdots \\
			y_{N-1} & \dots & y_{N-n} & u_N & \dots & u_{N-n}
		\end{bmatrix}
		\\ 
		\bm{p} &= \begin{bmatrix} -a_1 & \dots & -a_n & b_0 & \dots & b_n \end{bmatrix}^T
	\end{align*}

	\subsection{Output Error model}
	Lineární filtr používajíci minulé vstupy a výstupy modelu
	\begin{equation}
		\hat{y}(t) + a_1 \hat{y}(t-\Delta t) + \dots + a_n \hat{y}(t-n\Delta t)
		=
		b_0 u(t) + \dots + b_n u(t-n\Delta t)
	\end{equation}
	zbytek obdobně jako pro ARX jen se změnou $y$ na $\hat{y}$.


	\section{Popište postup identifikace lineárního modelu SISO systému ve tvaru FIR. }

	\begin{equation}
		\hat{y}(t) = g_0 u(t) + g_1 u(t-\Delta t) + \dots + g_n u(t-n\Delta t)
	\end{equation}

	\begin{align}
		\hat{y}_n &= g_0 u_n + g_1 u_{n-1} + \dots + g_n u_0 \\
		&\vdots \\
		\hat{y}_N &= g_0 u_{N} + g_1 u_{N-1} + \dots + g_n u_{N-n}
	\end{align}

	\begin{equation}
		\bm{\hat{y}} = \bm{\Phi} \bm{p}
		\;,\quad 
	\end{equation}
	kde
	\begin{align*}
		\bm{\hat{y}} &= \begin{bmatrix} \hat{y}_n & \dots & \hat{y}_N \end{bmatrix}^T
		\\ 
		\bm{\Phi}
		&=
		\begin{bmatrix}
			u_n & \dots & u_0 \\
			\vdots & \ddots & \vdots \\
			u_N & \dots & u_{N-n} 
		\end{bmatrix}
		\\ 
		\bm{p} &= \begin{bmatrix} g_0 & \dots & g_n \end{bmatrix}^T
	\end{align*}

	\section{Vysvětlete jak se sestavují Markovovy parametry a Hankelovy matice pro diskrétní stavový model. }

	\begin{align}
		\bm{x}_1 &= \bm{B} \bm{u}_0 & \bm{y}_0 &= \bm{D}\bm{u}_0 \\
		\bm{x}_2 &= \bm{A}\bm{B} \bm{u}_0 + \bm{B} \bm{u}_1 & \bm{y}_1 &= \bm{C}\bm{x}_1 + \bm{D} \bm{u}_1 \\
		\bm{x}_{k+1} &= \sum_{i=0}^{k} \bm{A}^{k-i} \bm{B} \bm{u}_i & \bm{y}_{k} &= \sum_{i=1}^k \underbrace{\bm{C} \bm{A}^{k-i} \bm{B}}_{h_i} \bm{u}_i + \underbrace{\bm{D}}_{h_0} \bm{u}_k 
	\end{align}
	\begin{equation}
		\bm{Y} = \bm{U} \bm{H}
	\end{equation}
	kde $\bm{Y}$ je matice tvořena výstupy, $\bm{U}$ matice vstupů a matice $\bm{H}$ \emph{Markovovy parametry}.
	\begin{align}
		\bm{Y}
		&=
		\begin{bmatrix}
			\bm{y}_0 & \bm{y}_0 & \dots & \bm{y}_q
		\end{bmatrix}
		\\
		\bm{U}
		&=
		\begin{bmatrix}
			\bm{u}_0 & \bm{u}_1 & \dots & \bm{u}_q \\
			\bm{0} & \bm{u}_0 & \dots & \bm{u}_{q-1} \\
			\vdots & & & \vdots \\
			\bm{0} & \dots & & \bm{u}_{q-p} 
		\end{bmatrix}
		\\
		\bm{H}
		&=
		\begin{bmatrix}
			\bm{h}_0 & \bm{h}_1 & \dots & \bm{h}_p
		\end{bmatrix}
	\end{align}

	\subsection{Hankelovy matice}
	\begin{equation}
		\bm{H}_1
		=
		\begin{bmatrix}
			h_1 & h_2 & \dots & h_p \\
			h_2 & h_3 & \dots & h_{p+1} \\
			\vdots & \vdots & & \vdots \\
			h_p & h_{p+1} & \dots & h_{2p-1}
		\end{bmatrix}
		\;,\quad 
		\bm{H}_2
		=
		\begin{bmatrix}
			h_2 & h_3 & \dots & h_{p+1} \\
			h_3 & h_4 & \dots & h_{p+2} \\
			\vdots & \vdots & & \vdots \\
			h_{p+1} & h_{p+2} & \dots & h_{2p}
		\end{bmatrix}
	\end{equation}

	\section{Popište identifikaci diskrétního stavového modelu pomocí metody ERA při znalosti Hankelových matic a Markovových parametrů. Co je to balancovaný tvar modelu ? }

	Hanklovy matice $\bm{H}_1$ a $\bm{H}_2$ lze zapsat jako
	\begin{equation}
		\bm{H}_1 = \bm{P}\bm{Q}
		\;,\quad 
		\bm{H}_2 = \bm{P}\bm{A}\bm{Q}
	\end{equation}
	kde $\bm{P}$ je matice pozorovatelnosti a $\bm{Q}$ matice říditelnosti.

	Balancovaný tvar zajistíme SVD rozkladem $\bm{H}_1$ na
	\begin{equation}
		\bm{H}_1 = \bm{V}\bm{\Gamma^2}\bm{U}^T
		\Rightarrow
		\bm{P} = \bm{V}\bm{\Gamma} \,,\; \bm{Q} = \bm{\Gamma}\bm{U}^T
	\end{equation}

	Matice identifikovaného systému pak jsou
	\begin{align}
		\bm{A} &= \bm{P}^+ \bm{H}_2 \bm{Q}^+ \\
		\bm{B} &= \bm{Q}[:,1:s] \\
		\bm{C} &= \bm{P}[1:r,:]
	\end{align}
	kde $s$ je počet vstupů, $r$ počet výstupů a $\bm{P}^+$, $\bm{Q}^+$ pseudo-inverze, které lze získat přímo z SVD rozkladu
	\begin{equation}
		\bm{P}^+ = \bm{\Gamma}^{-1}\bm{V}^T
		\,,\;
		\bm{Q}^+ = \bm{U}\bm{\Gamma}^{-1}
	\end{equation}

	\section{Vysvětlete rozdíl mezi modelem MDOF tlumeného mechanického systému s viskózním a strukturním tlumením. V čem je model se strukturním tlumením problematický pro časovou simulaci ? }

	\subsection*{Pohybové rovnice systému buzeného harmonickou funkcí}
	\begin{itemize}
		\item s viskózním tlumením
		\begin{equation}
			\bm{M}\bm{\ddot{x}} + \bm{B}\bm{\dot{x}} + \bm{K}\bm{x} = \bm{F}
		\end{equation}
		\item se strukturním tlumením
		\begin{equation}
			\bm{M}\bm{\ddot{x}} + (\bm{K} + j\bm{H}) \bm{x} = \bm{F}
		\end{equation}
	\end{itemize}

	Strukturní model tlumení se speciálně orientovaný na analýzu ve frekvenčí oblasti, jelikož v časové oblasti zanáší do simulace komplexní čísla.

	Ve frekvenční oblasti $\bm{B}\omega = \bm{H}$

	\section{Popište modální transformaci mechanického systému, vysvětlete pojem proporcionálního tlumení. }

	Pro systém ve tvaru
	\begin{equation*}
		\bm{M}\bm{\ddot{x}} + \bm{C}\bm{\dot{x}} + \bm{K}\bm{x} = \bm{F}
	\end{equation*}
	je základem modální trasformace nalezení řešení problému vlastních čísel
	\begin{equation*}
		\bm{K}\bm{V} = \bm{\Omega}^2 \bm{M} \bm{V}
	\end{equation*}
	kde $\bm{\phi}_i$ jsou vlastní vektory a $\Omega_i$ vlastní frekvence tvořící matice $\bm{V}$ a $\bm{\Omega}$  
	\begin{equation*}
		\bm{V} = \begin{bmatrix} \bm{\phi}_i & \dots & \bm{\phi}_N \end{bmatrix}
		\,,\;
		\bm{\Omega}^2 = \operatorname{diag}(\Omega_i^2)
		\;,\quad 
		i \in \langle 1,N \rangle
	\end{equation*}

	Po té platí
	\begin{equation*}
		\bm{V}^T\bm{M}\bm{V} = \bm{1}
		\;,\quad 
		\bm{V}^T\bm{K}\bm{V} = \bm{\Omega}^2
	\end{equation*}
	
	lze zavedením modální souřadnic $\bm{q} \,,\; \bm{x} = \bm{V}\bm{q}$ a vynásobením transponovanou maticí modální transformace $\bm{V}^T$ zleva, převést do tvaru
	\begin{equation*}
		\bm{I}\bm{\ddot{q}} + \bm{\Gamma}\bm{\dot{q}} + \bm{\Omega}^2 \bm{q} = \bm{V}^T \bm{F}
		\;,\quad 
		\bm{\Omega} = \bm{V}^T\bm{K}\bm{V}
		\,,\;
		\bm{\Gamma} = \bm{V}^T\bm{B}\bm{V}
	\end{equation*}

	O systému můžeme říct, že má proporční tlumení, je-li matice $\bm{\Gamma}$ diagonální s prvky $\beta_{ii} = 2 b_{r_i} \Omega_i$, kde $b_{r_i}$ jsou poměrné útlumy.

	Soustava se pak rozpadá na rovnice ve tvaru
	\begin{equation*}
		\ddot{q}_i + 2\,\Omega_i\xi_i \dot{q} + \Omega_i^2 q = f_i
		\;,\quad 
		f_i = \bm{\phi}_i \cdot \bm{F}
		\;,\quad 
		i \in \langle 1,N \rangle
	\end{equation*}

	\section{Napište a vysvětlete MAC kritérium pro porovnání vlastních tvarů modelu a vlastních tvarů naměřených. }
	\emph{Modal Assurance Criterion}
	\begin{equation}
	M_{rq} = \frac{\phi_{A_r}^T \phi_{X_r}}{(\phi_{A_r}^T\phi_{A_r})(\phi_{X_r}^T\phi_{X_r})}
	\;,\quad 
	r = 1,\dots,n \,,\; q = 1,\dots,m
	\end{equation}
	kde $n$ a $m$ je počet módů, $\phi_{A_q}$ měřené vlastní vekotry a $\phi_{X_r}$ vlastní vektory modelu.

	MAC kritérium se užívá pro zhodnocení shody vlastních vektorů modelu a měřeného systému. Při dokonalé shodě $\bm{M} = \bm{I}$.

	\section{Popište metodu SDOF identifikace mechanického systému z naměřených přenosových funkcí. }
	\section{Popište princip LSCF metody MDOF identifikace mechanického systému z naměřených přenosových funkcí. K čemu slouží stabilizační diagram ? }
	\emph{Least Squares Complex Frequency} domain estimator
	\begin{equation}
		\bm{G}(s) = \frac{B(s)}{A(s)}
	\end{equation}
	kde
	\begin{equation}
		\bm{B}
		=
		\begin{bmatrix}
			\bm{B}_{11}(s) & \bm{B}_{12}(s) & \dots & \bm{B}_{1N_i}(s) \\
			\bm{B}_{21}(s) & \bm{B}_{22}(s) & \dots & \bm{B}_{2N_i}(s) \\
			\vdots & \vdots & \ddots & \vdots \\
			\bm{B}_{N_o 1}(s) & \bm{B}_{N_o 2}(s) & \dots & \bm{B}_{N_o N_i}(s)
		\end{bmatrix}
		\;,\quad 
		\bm{A}
		=
		\begin{bmatrix}
			\bm{A}_{11}(s) & \bm{A}_{12}(s) & \dots & \bm{A}_{1N_i}(s) \\
			\bm{A}_{21}(s) & \bm{A}_{22}(s) & \dots & \bm{A}_{2N_i}(s) \\
			\vdots & \vdots & \ddots & \vdots \\
			\bm{A}_{N_o 1}(s) & \bm{A}_{N_o 2}(s) & \dots & \bm{A}_{N_o N_i}(s)
		\end{bmatrix}
	\end{equation}
	a
	\begin{equation}
	\bm{B}_{ii} = \sum_{j=0}^m b_{{ii}_j} s^j
	\;,\quad 
	\bm{A}_{ii} = \sum_{j=0}^n a_{{ii}_j} s^j
	\end{equation}

	Stabilizační diagram slouží k identifikaci fyzikálních pólů, které se nemění při postupném zvyšování řádu modelu.
	
	\section{Co je nelineární model Hammersteinova typu a nelineární model Wienerova typu. }

	Modely skládající se z dynamického lineárního přenosu $G(q)$ a statické nelineární funkce $f$
	
	\subsection*{Model Hammersteinova typu}	
	\begin{figure}[h!]
		\centering
		\includegraphics[width=.6\linewidth]{figs/HammersteinuvModel.pdf}
	\end{figure}
	
	\subsection*{Model Weinerova typu}	
	\begin{figure}[h!]
		\centering
		\includegraphics[width=.6\linewidth]{figs/WeineruvModel.pdf}
	\end{figure}

	\section{Uveďte strukturu nelineární identifikace používající koncept LOLIMOT. } \label{lolimot}
	\emph{LOcal LInear MOdels Tree}
	\begin{equation}
		\hat{y} = \sum_{i=1}^M \hat{y}_i \phi_i(\bm{u})
	\end{equation}
	kde $\hat{y}_i$ jsou \emph{lokální lineární modely} (typu FIR?)
	\begin{equation}
		\hat{y}_i = w_{i0} + w_{i1} u_1 + w_{i2} u_2 + \dots
	\end{equation}
	a $\phi_i(u)$ jejich platnostní funkce, které splňují
	\begin{equation}
		\sum_{i=1}^n \phi_i(u) = 1
		\;,\quad 
		\phi_i(u) = \frac{\mu_i(u)}{\sum_{i=1}^n \mu_i(u)}
		\;,\quad 
		\mu_i = \exp(-\frac{1}{2}(...))
	\end{equation}

	\begin{itemize}
	\item Vnitřní cyklus: výpočet koeficientů $w_i$ LLM při daném rozložení platnostních funkcí (exaktní - LSQ)
	\item Vnější cyklus: hledání vhodného "rozdělení" oblasti (heuristika)
	\end{itemize}


	\section{Uveďte postup identifikace nelineárního diskrétního dynamického modelu LOLIMOT typu NARX. }

	LOLIMOT typu NARX se liší od modelu v otázce \ref{lolimot} pouze ve tvaru lokálních modelů, který lze obecně zapsat jako
	\begin{equation}
		\hat{y}_k = f(u_k,u_{k-1},\dots,y_{k-1},y_{k-2},\dots)
	\end{equation}
	a způsobu jejich identifikace.

	\section{Vysvětlete pojmy testování a trénování identifikovaného modelu.}

	\subsection*{Trénování}
	Při trénování jsou parametry identifikovaného modelu upravovány tak, aby chování systému co nejvíce odpovídalo \emph{trénovacím datům}.

	\subsection*{Testování}
	Při testování porovnáváme chování identifikovaného modelu k nové sadě \emph{testovacích dat}, abychom ověřili, že nedošlo k přetrénovaní modelu, tzn. přizpůsobení ke konkrétním trénovacím datům, nikoliv obecnému chování.

	\section{Vysvětlete rozdíl mezi simulačním a predikčním trénováním a použitím LOLIMOT modelu a souvislost těchto pojmů s NARX a NOE modely. }

	\subsection*{Simulační trénování}
	Identifikovaný model je odpojený od reálného systému a pracuje pouze s jeho vstupy, kde chyba výstupu se kumuluje.

	Odpovídá schématu NOE
	\begin{equation}
		\hat{y}_k = f(u_k,u_{k-1},\dots,\hat{y}_{k-1},\hat{y}_{k-2},\dots)
	\end{equation}

	\subsection*{Predikční trénování}
	Model je připojen k výstupům modelu realného systému

	Odpovídá schématu NARX
	\begin{equation}
		\hat{y}_k = f(u_k,u_{k-1},\dots,y_{k-1},y_{k-2},\dots)
	\end{equation}

	\section{Uveďte příklad identifikace ad-hoc sestaveného dynamického modelu soustavy s pomocí obecných optimalizačních metod. }

	\section{Uveďte základní postup identifikace fyzikálního modelu získaného např. z MKP po transformaci do redukovaného modálního tvaru. V čem tato redukce usnadní postup identifikace ? }
	
	Na základě naměřených/navržených vlastností materiálu vytvořím mkp model ve tvaru
	\begin{equation*}
		\bm{M}\bm{\ddot{x}} + \bm{C}\bm{\dot{x}} + \bm{K}\bm{x} = \bm{F}
	\end{equation*}
	Ten následně transformuju do modálních souřadnic
	\begin{equation*}
		\bm{I}\bm{\ddot{q}} + \bm{\Gamma}\bm{\dot{q}} + \bm{\Omega}^2 \bm{q} = \bm{V}^T \bm{F}
		\;,\quad 
		\bm{\Omega} = \bm{V}^T\bm{K}\bm{V}
		\,,\;
		\bm{\Gamma} = \bm{V}^T\bm{B}\bm{V}
	\end{equation*}
	a oříznu mimodiagonální prvky transformované matice tlumení $\bm{\Gamma}$, přičemž prvky na diagonále lze vyjádřit jako $\beta_{ii} = 2 b_{r_i} \Omega_i$. Dále můžu provést redukci, ostraněním tvarů odpovídajících vyšším frekvencím systému.

	Vyslédkem je model závislý pouze na parametrech $\omega_i$, $b_{r_i}$ 	pro každý zanechaný tvar, které můžu dále optimalizovat. Oproti black-box identifikaci nehrozí vznik umělých artefaktů.

	\section{Uveďte příklad využití fenomenologického identifikovaného modelu pro simulaci. }
	
	Fenomenologicky například inentifikujeme tření
	\begin{figure}[h!]
		\centering
		\includegraphics[width=0.5\linewidth]{figs/Coulomb.png}
		\caption{Coulomb}
	\end{figure}
	\begin{figure}[h!]
		\centering
		\includegraphics[width=0.5\linewidth]{figs/Streibeck.png}
		\caption{Streibeck}
	\end{figure}
	nebo nelineární chování tlumiče.

	\section{Popište použití identifikovaného modelu soustavy v regulátoru s prediktivním řízením. Jak souvisí s pojmy NARX a NOE modelů ? }
% \end{multicols}
\end{document}